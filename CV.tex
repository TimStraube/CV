\documentclass[10pt, a4paper, sans]{moderncv}        
% possible options include font size ('10pt', '11pt' and '12pt'), paper size ('a4paper', 'letterpaper', 'a5paper', 'legalpaper', 'executivepaper' and 'landscape') and font family ('sans' and 'roman')

% moderncv themes
\moderncvstyle{banking}                             % style options are 'casual' (default), 'classic', 'banking', 'oldstyle' and 'fancy'
\moderncvcolor{green}                               % color options 'black', 'blue' (default), 'burgundy', 'green', 'grey', 'orange', 'purple' and 'red'
%\renewcommand{\familydefault}{\sfdefault}         % to set the default font; use '\sfdefault' for the default sans serif font, '\rmdefault' for the default roman one, or any tex font name
%\nopagenumbers{}                                  % uncomment to suppress automatic page numbering for CVs longer than one page

% adjust the page margins
\usepackage[scale=0.9]{geometry}
% \setlength{\footskip}{140.00005pt}    
%\setlength{\hintscolumnwidth}{3cm}                % if you want to change the width of the column with the dates
%\setlength{\makecvheadnamewidth}{10cm}            % for the 'classic' style, if you want to force the width allocated to your name and avoid line breaks. be careful though, the length is normally calculated to avoid any overlap with your personal info; use this at your own typographical risks...

% for luatex and xetex, do not use inputenc and fontenc
% see https://tex.stackexchange.com/a/496643
\ifxetexorluatex
  \usepackage{fontspec}
  \usepackage{unicode-math}
  \usepackage{hyperref}
  \defaultfontfeatures{Ligatures=TeX}
  \setmainfont{Latin Modern Roman}
  \setsansfont{Latin Modern Sans}
  \setmonofont{Latin Modern Mono}
  \setmathfont{Latin Modern Math} 
\else
  \usepackage[utf8]{inputenc}
  \usepackage[T1]{fontenc}
  \usepackage{lmodern}
\fi

\usepackage[ngerman]{babel}
\usepackage[utf8]{inputenc}

\title{Lebenslauf}        
\name{Tim Leonard}{Straube}
\social[linkedin]{tim-leonard-straube-48737a294}
\social[github]{TimStraube}       
\address{Sankt Stephansplatz 14}{78462}{Konstanz}
\email{tileone02@posteo.de} 
\renewcommand*{\bibliographyitemlabel}{[\arabic{enumiv}]}
\begin{document}
% \patchcmd{\makehead}{\\[2.5em]}{\hfill\raisebox{-1cm}[0cm][0cm]{\includegraphics[width=.13\textwidth]{Tim_Straube.png}}\\[0.0em]}{}{}

\makecvtitle

\vspace{-1cm}

\section{Projekte}
\subsection*{Quadstar, Bachelorarbeit}
Entwicklung eines Quadcoptersystems von Grund auf mit Python und C.
\subsection*{Alphazero}
Adaptation einer vorhandenen AlphaZero-Implementierung, auf Schiffe versenken mit Python und Javascript.
\subsection{Kommunikationstechnik}
Implementierung eines OFDM-Systems mit MATLAB.
\subsection{Verteilte Systeme}
Programmierung eines Mikrocontroller-Netzwerks zum Austausch von MQTT-basierten Sensorwerten, während gleichzeitig REST-Server als Benutzeroberflächen auf jedem Mikrocontroller gehostet werden mit C.
\subsection{SmartCalc.ModuleLayout und ModuleTEC Tecnomatix Modell, Projekte bei Fraunhofer}
Entwicklung einer Desktopapplikation zur Berechnung von Strom-Spannungs-Kennlinien von Photovoltaikmodulen aus Layout- und Topologiedaten sowie Berechnung optimaler Topologien und Zelllayouts mit MATLAB. Gefolgt von thermischen Simulationen und Produktionsmodellierung mit SimTalk.

\section{Erfahrung}
\cventry{2012 -- 2020}{Abitur}{Marie Curie Gymnasium}{Kirchzarten}{}{}
\cventry{Oktober 2020 -- Februar 2025}{Elektrotechnik und Informationstechnik B. Eng.}{Hochschule Konstanz}{Konstanz}{}{}
\cventry{März 2024 -- September 2024}{Tutor, Differenzialgleichungen}{Hochschule Konstanz}{Konstanz}{}{}
\cventry{Sommersemester 2023}{Praktikum, SmartCalc.ModuleLayout}{Fraunhofer Institut für Solare Energiesysteme}{Freiburg}{}{}
\cventry{September 2023 -- September 2024}{SmartCalc.ModuleLayout, ModuleTEC Tecnomatix Modell}{Fraunhofer Institut für Solare Energiesysteme}{Freiburg}{}{}
\cventry{Juli 2024 -- November 2024}{Quadstar, Bachelorarbeit}{Hochschule Konstanz}{Konstanz}{}{}

\section{Informationstechnik}
\subsection*{Programmiersprachen}
C, Python, MATLAB, SimTalk, Javascript
\subsection*{Betriebssysteme}
GNU/Linux, Windows, RTOS, Zephyr

\section{Sprachen}
\subsection{Deutsch}
Muttersprache
\subsection{English}
C1, Flüssig in Echtzeit

\section{Interests}
\subsection*{Mountainbiking}
Das ganze Jahr über
\subsection*{Flugsimulationen}
Seit Kindheitstagen

\end{document}
